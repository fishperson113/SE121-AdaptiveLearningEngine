\chapter{DEMO / PROTOTYPE MINH HỌA}

\section{Mục đích và phạm vi demo}
Sản phẩm được xây dựng là một \textbf{MVP (Minimum Viable Product)} nhằm mục đích kiểm chứng tính khả thi của mô hình Adaptive Learning Engine trên nền tảng web. Phạm vi của ứng dụng tập trung vào các tính năng cốt lõi nhất để vận hành thuật toán gợi ý và trực quan hóa dữ liệu năng lực, chưa bao gồm các tính năng quản lý lớp học phức tạp.

\section{Kiến trúc demo ở mức tổng quan}
Hệ thống được xây dựng theo kiến trúc 3 tầng hiện đại:
\begin{itemize}
    \item \textbf{Frontend:} ReactJS (Vite) cung cấp giao diện tương tác mượt mà.
    \item \textbf{Backend:} FastAPI (Python) đóng vai trò API Server, xử lý logic chọn câu hỏi và cập nhật Elo.
    \item \textbf{Database:} Sử dụng nền tảng \textbf{Supabase} (PostgreSQL) để quản lý dữ liệu người dùng và đồ thị kiến thức bền vững.
\end{itemize}

\section{Mô tả các chức năng chính của demo}
Dựa trên giao diện thực tế của ứng dụng, các chức năng chính được chia thành hai giao diện:

\subsection{Giao diện Học sinh (Student Dashboard)}
Giao diện dành cho học sinh tập trung vào việc hiển thị trạng thái năng lực và lộ trình cá nhân hóa:
\begin{itemize}
    \item \textbf{Hồ sơ năng lực:} Sử dụng biểu đồ tiến trình vòng tròn (Mastery Profile) để hiển thị mức độ bao phủ kiến thức.
    \item \textbf{Bản đồ học tập:} Trực quan hóa cây kiến thức và trạng thái từng concept.
    \item \textbf{Gợi ý thông minh:} Hiển thị câu hỏi được Engine lựa chọn dựa trên chiến lược Lowest\_Elo (ưu tiên vùng kiến thức yếu nhất trong tập sẵn sàng).
\end{itemize}

\begin{figure}[H]
    \centering
    \includegraphics[width=0.9\textwidth]{graphics/chapter-5/student_dashboard.png}
    \caption{Giao diện chính của Dashboard học sinh}
    \label{fig:student_db}
\end{figure}

\subsubsection{Giao diện làm bài tương tác}
Hình \ref{fig:student_quiz} minh họa quá trình hệ thống thích ứng trong thời gian thực:
\begin{itemize}
    \item \textbf{Adaptive Selection:} Câu hỏi được sinh ra (fetch) từ API dựa trên năng lực hiện tại.
    \item \textbf{Phản hồi tức thì:} Ngay sau khi học sinh submit, hệ thống tính toán xác suất đúng/sai và cập nhật Elo ngay lập tức.
\end{itemize}

\begin{figure}[H]
    \centering
    \includegraphics[width=0.9\textwidth]{graphics/chapter-5/student_question.png}
    \caption{Giao diện tương tác làm bài và phản hồi kết quả}
    \label{fig:student_quiz}
\end{figure}

\subsection{Giao diện Giáo viên (Teacher Analytics)}
Cung cấp cái nhìn tổng quát về tiến độ của cả lớp và từng cá nhân:
\begin{itemize}
    \item \textbf{Tổng quan lớp học:} Cho phép giáo viên theo dõi nhanh danh sách học sinh. 
    \item \textbf{Phân tích quỹ đạo (Learning Trajectory):} Biểu đồ trực quan hóa lộ trình học tập, giúp giáo viên dễ dàng theo dõi và đưa ra những nhận xét về xu hướng tiến bộ của học sinh.
    \item \textbf{Nhật ký tham số:} Dữ liệu audit chi tiết cho từng bước học tập.
\end{itemize}

\begin{figure}[H]
    \centering
    \includegraphics[width=0.9\textwidth]{graphics/chapter-5/teacher_dashboard.png}
    \caption{Giao diện Dashboard giáo viên và Phân tích quỹ đạo học tập}
    \label{fig:teacher_db}
\end{figure}

\section{Nhận xét và đánh giá}
Thông qua quá trình kiểm thử thủ công (Manual Testing), hệ thống MVP hoạt động ổn định và phản hồi chính xác. Các biểu đồ trực quan như Learning Trajectory giúp "số hóa" được quá trình nhận thức của học sinh, biến các con số thống kê khô khan thành thông tin hữu ích cho việc ra quyết định sư phạm.
