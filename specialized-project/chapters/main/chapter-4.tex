\chapter{THỰC NGHIỆM VÀ ĐÁNH GIÁ}

\section{Môi trường và công cụ thực nghiệm}
Quá trình thực nghiệm của dự án được chia làm hai giai đoạn rõ rệt nhằm đảm bảo tính chính xác của các tham số trước khi đưa vào ứng dụng thực tế.

Giai đoạn 1: \textbf{Mô phỏng (Simulation)}. Nhóm nghiên cứu sử dụng ngôn ngữ lập trình Python để xây dựng kịch bản mô phỏng thuật toán. Môi trường này cho phép chạy lặp lại quá trình học của hàng nghìn học sinh giả định để định chuẩn các tham số cốt lõi.

Giai đoạn 2: \textbf{Ứng dụng thử nghiệm (Application Testing)}. Hệ thống được triển khai dưới dạng ứng dụng MVP (Minimum Viable Product Full-stack) sử dụng:
\begin{itemize}
    \item \textbf{Backend:} Python FastAPI xử lý logic thích ứng.
    \item \textbf{Database:} Nền tảng Supabase (PostgreSQL) lưu trữ dữ liệu.
    \item \textbf{Frontend:} Giao diện tương tác người dùng ReactJS.
\end{itemize}

\section{Dữ liệu và kịch bản thử nghiệm}
Để đánh giá hiệu quả của hệ thống, nhóm nghiên cứu thiết lập các thành phần dữ liệu và kịch bản như sau:

\subsection{Hồ sơ Người học (Profiles)}
Ba nhóm hồ sơ tiêu biểu được thiết lập để đại diện cho các đối tượng học sinh khác nhau:
\begin{itemize}
    \item \textbf{Học sinh Giỏi đều:} Có xác suất trả lời đúng cao ở hầu hết các concept.
    \item \textbf{Học sinh Yếu đều:} Có năng lực thấp đồng đều ở mọi chương.
    \item \textbf{Học sinh Yếu Hàm số:} Có năng lực trung bình nhưng hổng kiến thức nghiêm trọng tại chương Hàm số.
\end{itemize}

\subsection{Kịch bản Phân tích Độ nhạy (Sensitivity Analysis)}
Trên môi trường mô phỏng, hệ thống thực hiện chạy lưới tham số (Grid Search) để tìm ra cấu hình tối ưu, bao gồm việc thay đổi Ngưỡng thành thạo (Mastery Threshold) và Hệ số K.

\subsection{Chiến lược gợi ý}
Hai chiến lược chính được đưa vào so sánh:
\begin{itemize}
    \item \textbf{Chiến lược Lowest\_Elo:} Ưu tiên gợi ý các concept mà học sinh có điểm năng lực thấp nhất trong tập sẵn sàng (Ready Set).
    \item \textbf{Chiến lược Cross\_Chapter:} Ưu tiên mở khóa các chương mới thông qua các cạnh liên kết liên chương.
\end{itemize}

\section{Cách thức triển khai thực nghiệm}
Đối với phần mô phỏng, thuật toán sẽ tự động chạy qua 500 bước học tập cho từng hồ sơ để ghi nhận sự biến thiên của Elo.

Đối với phần ứng dụng ứng dụng thực tế (MVP), nhóm thực hiện kiểm thử thủ công (Manual Testing) bằng cách đóng vai các profile học sinh, thực hiện làm bài trên giao diện web để xác minh tính đúng đắn của logic gợi ý và khả năng phản hồi thời gian thực của hệ thống.

\section{Kết quả thực nghiệm}
Dựa trên kết quả mô phỏng và kiểm thử, nhóm nghiên cứu đã chốt lại cấu hình Engine cuối cùng như sau:

\subsection{Cấu hình Engine tối ưu (Final Configuration)}
Dựa trên kết quả thực nghiệm, nhóm nghiên cứu đã xác định bộ tham số hoạt động ổn định nhất:
\begin{itemize}
    \item \textbf{Hệ số K = 24 (Constant):} Mang lại sự cân bằng, giúp biểu đồ mượt mà và phản ánh chính xác quá trình tích lũy kiến thức.
    \item \textbf{Ngưỡng thành thạo = 1250:} Điểm xác nhận nắm vững kiến thức tối ưu.
    \item \textbf{Chiến lược Lowest\_Elo:} Giúp củng cố vùng kiến thức yếu nhất, hạn chế "vỡ lộ trình" (số lần nhảy chương giảm từ ~26 xuống ~7 lần ở học sinh yếu đều).
\end{itemize}

\begin{figure}[htbp]
    \centering
    \includegraphics[width=0.8\textwidth]{graphics/chapter-4/Strategy Comparison – lowest_elo vs cross_chapter_unlock.png}
    \caption{So sánh Chapter Switches giữa các chiến lược gợi ý}
    \label{fig:strategy_impact}
\end{figure}

\section{Phân tích và thảo luận kết quả}
Với cấu hình tối ưu, hệ thống thể hiện khả năng cải thiện lộ trình qua hai khía cạnh:

\subsection{Giảm thiểu hiện tượng "Zigzag giả"}
Nghiên cứu phân biệt giữa Zigzag "tốt" (khám phá kiến thức) và Zigzag "xấu" (nhiễu tín hiệu). Đồ thị Advanced giúp giảm mạnh hiện tượng Zigzag "xấu":
\begin{itemize}
    \item \textbf{Học sinh Yếu đều:} Giảm 50\% (từ 14 xuống 7 lần nhảy chương).
    \item \textbf{Học sinh Yếu Hàm số:} Giảm 90\% (từ 29 xuống 3 lần), neo học sinh vào đúng lỗ hổng cần lấp.
\end{itemize}
Hình \ref{fig:zigzag_compare} cho thấy rõ sự khác biệt này. Trong khi Baseline (màu cam) có số lần chuyển chương cao bất thường do thiếu ràng buộc chặt chẽ, Advanced graph (màu xanh) duy trì được sự ổn định cần thiết.

\begin{figure}[htbp]
    \centering
    \includegraphics[width=0.8\textwidth]{graphics/chapter-4/Chapter Switches – Baseline vs Advanced.png}
    \caption{So sánh hiện tượng Zigzag/Nhảy chương giữa Baseline và Advanced}
    \label{fig:zigzag_compare}
\end{figure}

\subsection{Kiểm soát kích thước Ready Set}
Đồ thị Advanced đảm bảo quy mô tập sẵn sàng ổn định, giúp tập trung vào Vùng phát triển gần nhất.
Hình \ref{fig:ready_set_size} minh họa phân phối kích thước tập sẵn sàng (Ready Set Size). Chiến lược Lowest\_Elo giữ kích thước này ở mức thấp và ổn định (dao động thấp), giúp học sinh tập trung vào một nhóm nhỏ các khái niệm phù hợp nhất thay vì bị phân tán bởi quá nhiều lựa chọn như các chiến lược khác.

\begin{figure}[htbp]
    \centering
    \includegraphics[width=0.8\textwidth]{graphics/chapter-4/Ready Set Size Distribution.png}
    \caption{Phân phối kích thước Ready Set theo chiến lược}
    \label{fig:ready_set_size}
\end{figure}
