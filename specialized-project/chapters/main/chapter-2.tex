\chapter{TỔNG QUAN NGHIÊN CỨU VÀ CƠ SỞ LÝ THUYẾT}

\section{Tổng quan hướng nghiên cứu liên quan}
Trong lĩnh vực học tập thích ứng, nhiều nghiên cứu tập trung vào việc cá nhân hóa lịch ôn tập và trình tự học dựa trên hành vi ghi nhớ, tiêu biểu là các mô hình spaced repetition. Các cách tiếp cận này được áp dụng rộng rãi trong các lĩnh vực như học từ vựng, ngoại ngữ hoặc các bài học mang tính ghi nhớ, nơi kiến thức có thể được chia nhỏ thành các đơn vị tương đối độc lập.

Tuy nhiên, đối với các môn học có cấu trúc kiến thức phụ thuộc chặt chẽ như Toán học, cách tiếp cận trên bộc lộ nhiều hạn chế. Nhiều hệ thống chưa xuất phát từ việc xác định rõ các mối quan hệ kiến thức tiên quyết, mà chủ yếu điều chỉnh tần suất hoặc thứ tự xuất hiện của nội dung học. Việc thiếu một mô hình biểu diễn tường minh cấu trúc kiến thức khiến hệ thống khó phát hiện chính xác nguyên nhân dẫn đến việc học sinh gặp khó khăn ở một nội dung cụ thể.

Một số hướng nghiên cứu gần đây bắt đầu quan tâm đến việc biểu diễn kiến thức dưới dạng đồ thị (concept graph hoặc knowledge graph) nhằm phản ánh mối quan hệ phụ thuộc giữa các khái niệm. Tuy vậy, việc xây dựng đồ thị kiến thức cụ thể cho môn Toán THPT, cũng như khai thác đồ thị này để tạo ra lộ trình học tập phân hóa rõ ràng giữa các nhóm năng lực khác nhau, vẫn chưa được nghiên cứu một cách hệ thống.

\section{Các khái niệm và nền tảng lý thuyết}
\blindtext

\section{Các phương pháp / mô hình hiện có}
\blindtext

\section{So sánh và đánh giá các hướng tiếp cận}
\blindtext

\section{Khoảng trống nghiên cứu}
\blindtext
